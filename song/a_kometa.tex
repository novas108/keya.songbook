\begin{song}{Kometa}{C}{Jaromír Nohavica}

\begin{SBVerse}

\Ch{ami}{Spatřil} jsem kometu, oblohou letěla,

chtěl jsem jí zazpívat, ona mi zmizela,

\Ch{dmi}{zmizela} jako laň \Ch{G7}{u lesa} v remízku,

\Ch{C}{v }očích mi zbylo jen \Ch{E7}{pár} žlutých penízků.

\end{SBVerse}

\begin{SBVerse}

Penízky ukryl jsem do hlíny pod dubem,

až příště přiletí, my už tu nebudem,

my už tu nebudem, ach, pýcho marnivá,

spatřil jsem kometu, chtěl jsem jí zazpívat.

\end{SBVerse}

\begin{SBChorus}

\Ch{ami}{O vodě}, o trávě, \Ch{dmi}{o lese,} \Ch{G7}{o }smrti, se kterou smířit \Ch{C}{nejde} se,

\Ch{ami}{o }lásce, o zradě, \Ch{dmi}{o svě}tě

\Ch{E}{a o }všech lidech, co kdy \Ch{E7}{žili} na téhle \Ch{ami}{planetě.}

\end{SBChorus}

\begin{SBVerse}

Na hvězdném nádraží cinkají vagóny,

pan Kepler rozepsal nebeské zákony,

hledal, až nalezl v hvězdářských triedrech

tajemství, která teď neseme na bedrech.

\end{SBVerse}

\begin{SBVerse}

Velká a odvěká tajemství přírody,

že jenom z člověka člověk se narodí,

že kořen s větvemi ve strom se spojuje

a krev našich nadějí vesmírem putuje.

\end{SBVerse}

\begin{SBChorus}

Na na na ...

\end{SBChorus}

\begin{SBVerse}

Spatřil jsem kometu, byla jak reliéf

zpod rukou umělce, který už nežije,

šplhal jsem do nebe, chtěl jsem ji osahat,

marnost mne vysvlékla celého donaha.

\end{SBVerse}

\begin{SBVerse}

Jak socha Davida z bílého mramoru

stál jsem a hleděl jsem, hleděl jsem nahoru,

až příště přiletí, ach, pýcho marnivá,

my už tu nebudem, ale jiný jí zazpívá.

\end{SBVerse}

\begin{SBChorus}

O vodě, o trávě, o lese, o smrti, se kterou smířit nejde se,

o lásce, o zradě, o světě, bude to písnička o nás a kometě.

\end{SBChorus}

\end{song}

\pagebreak