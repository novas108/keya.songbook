\setcounter{page}{57}
\begin{song}{Nosorožec}{C}{Karel Plíhal}

\begin{SBVerse}

1. \Ch{ami}{Přivedl} jsem domů Božce \Ch{dmi}{nádhernýho} \Ch{ami}{nosorožce,}

   \Ch{dmi}{originál} \Ch{ami}{tlustokožce,} \Ch{D7}{koupil} jsem ho v h\Ch{E}{ospodě.}

   \Ch{ami}{Za dva} rumy a dvě vodky př\Ch{dmi}{ipadal} mi v\Ch{ami}{elmi} krotký,

   \Ch{dmi}{pošlapal} mi p\Ch{ami}{olobotky,} \Ch{E}{ale} jinak v p\Ch{ami}{ohodě.}

   \Ch{dmi}{Vznikly} menší p\Ch{ami}{otíže} př\Ch{E}{i }nástupu d\Ch{ami}{o zdviže,}

   \Ch{dmi}{při }výstupu \Ch{ami}{ze }zdviže \Ch{D7}{už nám} to šlo l\Ch{E}{ehce.}

   \Ch{ami}{Vzikly} větší potíže, \Ch{dmi}{když} Božena v \Ch{ami}{negližé,}

   \Ch{dmi}{když} Božena v \Ch{ami}{negližé} \Ch{E}{řvala,} že ho \Ch{ami}{nechce.}

   \end{SBVerse}

\begin{SBVerse}

Marně jsem se snažil Božce vnutit toho tlustokožce,

   originál nosorožce, co nevidíš v obchodech.

   Řvala na mě, že jsem bohém, pak mi řekla padej sbohem,

   zabouchla před nosorohem, tak tu sedím na schodech.

   Co nevidím - souseda, jak táhne domů medvěda,

   originál medvěda, tuším značky grizzly.

   Už ho ženě vnucuje a už ho taky pucuje

   a zamčela a trucuje, tak si to taky slízli.

\end{SBVerse}

\begin{SBVerse}

\Ch{ami}{Tak }tu sedím se sousedem, s \Ch{dmi}{nosorohem} \Ch{ami}{a s }medvědem,

   \Ch{dmi}{nadáváme} \Ch{ami}{jako} jeden \Ch{H7}{na }ty naše \Ch{emi}{slepice.}

   \end{SBVerse}

   \end{song}

\pagebreak
