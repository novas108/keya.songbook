\setcounter{page}{64}
\begin{song}{Pánové nahoře}{C}{Jaromír Nohavica}

\begin{SBVerse}

1. \Ch{C}{Pánové} naho\Ch{emi}{ře,} já \Ch{dmi7}{píšu} vám dnes \Ch{G7}{psaní}

a \Ch{dmi7}{nevím} vlastně \Ch{C}{ani,} bu\Ch{dmi7}{dete-li} ho \Ch{G7}{číst,}

\Ch{C}{přišlo} mi ve \Ch{emi}{středu} do \Ch{dmi7}{války} předvo\Ch{G7}{lání,}

je \Ch{dmi7}{to bez} odvo\Ch{C}{lání, }tím \Ch{dmi7}{prý} si \Ch{G7}{mám} být \Ch{C}{jist.}

\Ch{F}{Pánové} nahoře, \Ch{Cdim}{já už }to lejstro \Ch{emi}{spálil,}\footnote{Cdim lze nahradit D7?}

už \Ch{Ami}{jsem} si kufry \Ch{dmi7}{sbalil,} správcové vrátil \Ch{G7}{klíč,}

\Ch{C}{pánové} \Ch{emi}{nahoře,} uc\Ch{dmi7}{tivě} se vám \Ch{G7}{klaním}

a \Ch{dmi7}{zítra} vlakem \Ch{C}{ranním} \Ch{dmi7}{odjíždím} \Ch{G7}{někam} \Ch{C}{pry}\Ch{emi}{č.} \Ch{dmi7}{} \Ch{G7}{}
\end{SBVerse}

\begin{SBVerse}
Co už jsem na světě, viděl jsem zoufat matky

nad syny, kteří zpátky se nikdy nevrátí,

slyšel jsem dětský pláč a viděl jejich slzy,

které snadno a brzy se z očí neztratí.

Znám vaše věznice, znám vaše kriminály,

i ty, kterým jste vzali život či minulost,

vím také, že máte solidní arzenály,

i když jste povídali o míru víc než dost.
\end{SBVerse}

\begin{SBVerse}
Pánové nahoře, říká se, že jste velcí

a na věci to přece vůbec nic nemění,

pánové nahoře, na to jste vážně malí,

abyste vydávali rozkazy k vraždění.

Musí-li války být, běžte si válčit sami

a vaši věrní s vámi, mě, mě nechte být,

jestli mě najdete, můžete klidní být,

střílejte, neváhejte, já zbraň nebudu mít\dots
\end{SBVerse}

\end{song}

\pagebreak
