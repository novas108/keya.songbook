\begin{song}{Pískající cikán}{G}{Spiritual kvintet}

\begin{SBVerse}

\Ch{G}{Dívka} \Ch{ami}{loudá} se \Ch{G}{vinicí}\Ch{ami}{,}\Ch{G}{tam,} kde \Ch{ami}{zídka} je \Ch{hmi}{nízk}\Ch{ami}{á,}

\Ch{G}{tam,} kde \Ch{ami}{stráň} končí \Ch{hmi}{vonící,}\Ch{C}{tam} \Ch{G}{písnič}\Ch{C}{ku} někdo \Ch{G}{pí}\Ch{C}{sk}\Ch{D}{á.}

\end{SBVerse}

\begin{SBVerse}

Ohlédne se a "Propána!", v stínu, kde stojí líska,

švárného vidí cikána, jak leží, písničku píská.

\end{SBVerse}

\begin{SBVerse}

Chvíli tam stojí potichu, písnička si jí získá,

domů jdou spolu do smíchu, je slyšet cikán, jak píská.

\end{SBVerse}

\begin{SBVerse}

Jenže tatík, jak vidí cikána, pěstí do stolu tříská,

"Ať táhne pryč, vesta odraná, groš nemá, něco ti spíská."

\end{SBVerse}

\begin{SBVerse}

Teď smutnou dceru má u vrátek, jen Bůh ví, jak se jí stýská,

"kéž vrátí se mi zas nazpátek ten který v dálce si píská."

\end{SBVerse}

\begin{SBVerse}

Pár šidel honí se po louce, v trávě rosa se blýská,

cikán, rozmarýn v klobouce, jde dál a písničku píská.

\end{SBVerse}

\begin{SBVerse}

Na závěr zbývá už jenom říct, v čem je ten kousek štístka:

peníze často nejsou nic, má víc, kdo po svém si píská ...

\end{SBVerse}

\end{song}

\pagebreak