\setcounter{page}{89}
\begin{song}{Topoly}{C}{Ondřej Balík}
\begin{footnotesize}
\begin{SBChorus*}
\Ch{hmi}{Jako} ty topoly máme svý stíny, žijeme ze soli vody a hlíny.
\end{SBChorus*}
\begin{SBVerse}
\Ch{ami}{Jako} ty topoly \Ch{dmi7}{stojíme} vzpříma, \Ch{Cmaj}{dokud} nám dovolí s\Ch{F}{lunce} a z\Ch{F}{im}\Ch{E}{a},

\Ch{ami}{jako} ty topoly \Ch{dmi7}{celí se} třesem, \Ch{Cmaj}{sekyra} zabolí, s\Ch{hmi7}{ekyra} zabolí,

\Ch{F}{znamení} \Ch{F}{ne}\Ch{E}{sem}.

\Ch{ami}{Jako} ty topoly, \Ch{dmi7}{když} nemáš sílu, \Ch{Cmaj}{tak} lidi z okolí \Ch{F}{přinesou} \Ch{F}{pil}\Ch{E}{u}.

\Ch{ami}{Jako} ty topoly \Ch{dmi7}{v} pláči a křiku, \Ch{Cmaj}{bouře} tě oholí, \Ch{hmi7}{bouře} tě oholí

\Ch{G4}{do} vykřiční\Ch{F#4}{ku}.
\end{SBVerse}
\begin{SBVerse}
Jako ty topoly strach někde nízko v nejhlubším údolí k nebi máš blízko.

Jako ty topoly horko nás moří, belháš se o holi, belháš se o holi,

za tebou hoří.

Jako ty topoly až přejde září, strniště na poli, listí a stáří.

Jako ty topoly do příští zimy, koho to zabolí, koho to zabolí,

budou tu jiní.
%sloka 2
\end{SBVerse}
[:h03e232h3e012:][hmi G A f\#mi]
\begin{SBVerse}
Jako ty topoly, jako ty stromy,

jeden si dovolí, jinej se zlomí.

Jako ty topoly máme svý stíny,

žijeme ze soli, vody a hlíny.

Jako ty...
%sloka 3
\end{SBVerse}
\end{footnotesize}
\end{song}

\pagebreak
