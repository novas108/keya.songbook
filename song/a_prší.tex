\setcounter{page}{71}
\begin{song}{Prší}{C}{Karel Plíhal}

\begin{SBChorus}

\Ch{C}{P}rš\Ch{C/H}{í} \Ch{dmi}{a hvězdy} \Ch{G}{na} plakátech \Ch{C}{b}lednou\Ch{C/H}{,}

\Ch{dmi}{zpívám} si \Ch{E}{spolu} s repro\Ch{ami}{bednou}\Ch{ami/G}{,}

jak ta \Ch{F}{láska} deštěm \Ch{G}{voní},

stejně \Ch{C}{voněla} i \Ch{G}{loni}, zkrátka

\end{SBChorus}

\begin{SBVerse}

\Ch{C}{P}rš\Ch{C/H}{í} \Ch{dmi}{a soused} \Ch{G}{chodí} sadem s \Ch{C}{konví}\Ch{C/H}{,}

\Ch{dmi}{každej} se \Ch{E}{diví,} jenom \Ch{ami}{on ví}\Ch{ami/G}{,}

proč \Ch{F}{místo} toho \Ch{E}{kropení} si \Ch{A}{neza}leze k \Ch{dmi}{topení}

a \Ch{G}{nepřečte} si \Ch{C}{M}cBaina, proč \Ch{F}{voz}í mouku \Ch{E}{do} mlejna.

\end{SBVerse}

\begin{SBChorus}

\end{SBChorus}

\begin{SBVerse}

Prší a soused venku prádlo věší,

práce ho, jak je vidět, těší,

ač promáčen je na nitku, tak na co volat sanitku,

stejně na čísle blázince je věčně někdo na lince.

\end{SBVerse}

\end{song}

\clearpage
