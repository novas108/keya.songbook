\setcounter{page}{15}
\begin{song}{Černá díra}{G}{Karel Plíhal}

\begin{SBVerse}

\Ch{G}{Mí}vali jsme \Ch{D}{dě}dečka, \Ch{C}{sta}rého už \Ch{G}{pá}na,

stalo se to v \Ch{D}{čer}venci \Ch{C}{jed}nou časně \Ch{D}{zrá}\Ch{G}{na},

\Ch{emi}{šel} do sklepa \Ch{C}{pro} vidle, \Ch{A}{aby} seno \Ch{D}{sklí}zel,

\Ch{G}{už} se ale \Ch{D}{ne}vrátil, \Ch{C}{pro}stě někam \Ch{D}{zmi}\Ch{G}{zel}.

\end{SBVerse}

\begin{SBVerse}

Máme doma ve sklepě malou černou díru,

na co přijde, sežere, v ničem nezná míru,

nechoď, babi, pro uhlí, sežere i tebe,

už tě nikdy nenajdou příslušníci VB.

\end{SBVerse}

\begin{SBVerse}

Přišli vědci zdaleka, přišli vědci zblízka,

babička je nervózní a nás, děti, tříská,

sama musí poklízet, běhat kolem plotny,

a děda je ve sklepě nekonečně hmotný.

\end{SBVerse}

\begin{SBVerse}

Hele, babi, nezoufej, moje žena vaří

a jídlo se jí většinou nikdy nepodaří,

půjdu díru nakrmit zbytky od oběda,

díra všechno vyvrhne, i našeho děda.

\end{SBVerse}

\begin{SBVerse}

Tak jsem díru nakrmil zbytky od oběda,

díra všechno vyvrhla, i našeho děda,

potom jsem ji rozkrájel motorovou pilou,

opět člověk zvítězil nad neznámou silou.

\end{SBVerse}

\begin{SBVerse}

\Ch{A}{Dědeček} se \Ch{E}{raduje,} \Ch{D}{že je }zase v \Ch{A}{penzi,}

teď je naše p\Ch{E}{ísnička} zr\Ch{D}{alá pro} rec\Ch{E}{enz}\Ch{A}{i.}

\end{SBVerse}

\end{song}
\clearpage
