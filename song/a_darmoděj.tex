\setcounter{page}{18}
\begin{song}{Darmodej}{C}{Jaromír Nohavica}
  \begin{SBVerse}
\Ch{ami}{Šel} včera městem \Ch{emi}{muž} a šel po hlavní \Ch{ami}{třídě,} \Ch{emi}{ }

\Ch{ami}{šel} včera městem \Ch{emi}{muž} a já ho z okna \Ch{ami}{viděl}\Ch{emi}{, }

\Ch{C}{na} flétnu chorál \Ch{G}{hrá}l, znělo to jako \Ch{ami}{zvon}

a byl v tom všechen \Ch{emi}{žal,} ten krásný dlouhý \Ch{F}{tón,}

a já jsem náhle \Ch{F#dim}{věděl:} Ano, to je \Ch{E7}{on}, to je \Ch{ami}{on.}
\footnote{F\#dim lze nahradit D7}
  \end{SBVerse}
  \begin{SBVerse}
Vyběh' jsem do ulic jen v noční košili, 

v odpadcích z popelnic krysy se honily

a v teplých postelích lásky i nelásky 

tiše se vrtěly rodinné obrázky,

a já chtěl odpověď na svoje otázky, otázky.
  \end{SBVerse}
\begin{SBChorus}
$|$:\Ch{Ami}{Na,} nana\Ch{Emi}{na }nana\Ch{C}{na} nana \Ch{G}{nananána,}

\Ch{Ami}{na }nana\Ch{F}{na} nana\Ch{F#dim}{na nana} \Ch{E7}{nananána.} :$|$
\end{SBChorus}
\begin{SBVerse}
Dohnal jsem toho muže a chytl za kabát,

měl kabát z hadí kůže, šel z něho divný chlad,

a on se otočil, a oči plné vran,

a jizvy u očí, celý byl pobodán,

a já jsem náhle věděl kdo je onen pán onen pán

  \end{SBVerse}
\begin{SBVerse}

Celý se strachem chvěl,když jsem tak k němu došel,

a v ústech flétnu měl od Hieronyma Bosche,

stál měsíc nad domy jak čírka ve vodě,

jak moje svědomí, když zvrací v záchodě,

a já jsem náhle věděl: to je Darmoděj, můj Darmoděj.

  \end{SBVerse}
\begin{SBChorus}
Můj Darmoděj, vagabund osudů a lásek,

jenž prochází všemi sny,ale dnům vyhýbá se.

Můj Darmoděj, krásné zlo, jed má pod jazykem,

když prodává po domech jehly se slovníkem.

\end{SBChorus}
\begin{SBVerse}
Šel včera městem muž, podomní obchodník,

šel, ale nejde už, krev skápla na chodník,

já jeho flétnu vzal a zněla jako zvon

a byl v tom všechen žal, ten krásný dlouhý tón,

a já jsem náhle věděl: ano, já jsem on, já jsem:

  \end{SBVerse}
  
\begin{SBChorus}
Váš Darmoděj \dots
\end{SBChorus}
 \end{song}
\pagebreak
