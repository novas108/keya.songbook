\setcounter{page}{56}
\begin{song}{Nagasaki Hirošima}{C}{Karel Plíhal}

\begin{SBVerse}

\Ch{C}{Tram}vají \Ch{G}{dvoj}kou \Ch{F}{jez}díval jsem \Ch{G}{do Žide}\Ch{C}{nic,}  \Ch{G}{}  \Ch{F}{}  \Ch{G}{}

z \Ch{C}{tak} velký \Ch{G}{lá}sky \Ch{F}{vět}šinou \Ch{G}{ne}zbyde \Ch{ami}{nic},

Z \Ch{F}{ta}kový \Ch{C}{lá}sky \Ch{F}{jsou} kruhy \Ch{C}{pod} oči\Ch{G}{ma}

a dvě \Ch{C}{spá}lený \Ch{G}{srd}ce - \Ch{F}{Na}gasaki, \Ch{G}{Hi}roši\Ch{C}{ma. } \Ch{G}{ } \Ch{F}{ } \Ch{G}{ }

\end{SBVerse}

\begin{SBVerse}

Jsou jistý věci, co bych tesal do kamene,

tam, kde je láska, tam je všechno dovolené,

a tam, kde není, tam mě to nezajímá,

jó, dvě spálený srdce - Nagasaki, Hirošima.

\end{SBVerse}

\begin{SBVerse}

Já nejsem svatej, ani ty nejsi svatá,

ale jablka z ráje bejvala jedovatá,

jenže hezky jsi hřála, když mi někdy bylo zima,

jó, dvě spálený srdce - Nagasaki, Hirošima.

\end{SBVerse}

\begin{SBVerse} = 1.
\end{SBVerse}
\begin{SBVerse}
Mladičká  básnířka s  korálky nad kotníky, 

bouchala  na dvířka paláce  poetiky,

s někým se  vyspala, někomu  nedala, láska jako hobby,

tak o tom napsala sonet na čtyři doby.    
\end{SBVerse}

\begin{SBVerse}
Své srdce skloňovala podle vzoru Felinghetti,

ve vzduchu nechávala viset vždy jen půlku věty,

plná tragiky, plná mystiky, plná splínu,

pak jí to otiskli v jednom magazínu.
\end{SBVerse}

\begin{SBVerse}
Bývala viděna v malém baru u rozhlasu,

od sebe kolena a cizí ruka kolem pasu,

trochu se napila, trochu se opila na účet redaktora

a týden nato byla hvězdou Mikrofóra.
\end{SBVerse}

\begin{SBVerse}
Pod paží nosila rozepsané rukopisy,

ráno se budila vedle záchodové mísy,

múzou políbená, životem potřísněná, plná zázraků

a pak ji vyhodili z gymplu i z baráku.
\end{SBVerse}

\begin{SBVerse}
Šly řeči v okolí, že měla něco s esenbáky,

ať bylo cokoli, přestala věřit na zázraky,

cítila u srdce, jak po ní přešla železná bota,

tak o tom napsala sonet ze života.
\end{SBVerse}

\begin{SBVerse}
Pak jednou v pondělí přišla na koncert na koleje

a hlasem pokorným prosila o text Darmoděje,

kterého ho vzala, pak se dala tichounce do pláče.

a její slzy kapaly na její mrkváče,    
\end{SBVerse}
\end{song}

\clearpage
