\setcounter{page}{28}
\begin{song}{Hospoda U Davida}{G}{Fleret}
\begin{SBVerse}
V \Ch{C}{hospodě} \Ch{G}{na} náměs\Ch{C}{tí} může\Ch{G}{te} při troše \Ch{ami}{štěs}tí

sehnat \Ch{D}{místo} u sto\Ch{G}{lu,}

sednout si na lavi\Ch{C}{ci,} dát si \Ch{G}{pravou} kyseli\Ch{ami}{ci}

anebo \Ch{D}{rum} a kofo\Ch{G}{lu}.
\end{SBVerse}
\begin{SBVerse}
Říká se tam "U Davida", klasická třetí třída,

zákaz her s výjimkou domina,

a když je někdy pod psa venku, 

hospodský též dá si sklenku,

přisedne \Ch{D}{k }vám \Ch{G}{a }vzpomí\Ch{C}{ná}.
\end{SBVerse}
\begin{SBChorus}
\Ch{H}{Ač} je to téměř k nevíře, já \Ch{emi}{býval} slavným kumštýřem,

\Ch{H}{zpíval} a napříč flétnou \Ch{emi}{hrál},

\Ch{H}{procestoval} kraje cizí, v \Ch{emi}{rozhlase} a v \Ch{A}{televizi}

\Ch{C}{každej} mě \Ch{D}{znal},

\Ch{H}{a }holky krásný jako břízky \Ch{emi}{nosily} mi z domu řízky,

\Ch{H}{po }koncertě rovnou do šat\Ch{emi}{ny, }

\Ch{H}{za }podpisy do cancáku \Ch{emi}{vlezly} mi až \Ch{A}{do} spacáku,

\Ch{C}{ale }to je \Ch{D}{dnes} už neplat\Ch{G}{ný.}
\end{SBChorus}

\begin{SBVerse}
Pak se napije a kývá hlavou, jako by přemýšlel nad zašlou slávou,

a je ticho a jen pendlovky jdou,

pak vstane a jde za své pípy, a za chvíli už zas vtipy

rozléhají se hospodou.
\end{SBVerse}
\begin{SBVerse}
A když nese další várku jak Děda Mráz s nůší dárků,

štamgastům se oči rozzáří,

co tam po odjezdu vlaků, hned si dají po panáku,

tržba se dnes jistě vydaří.
\end{SBVerse}
\begin{SBChorus}
Jó, je to dnes až k nevíře, ten chlap býval slavným kumštýřem,

zpíval a napříč flétnou hrál,

procestoval kraje cizí, v rozhlase a v televizi

každej ho znal.



A teď tady pivo točí a dojetím mu vlhnou oči,

když sem přijdou kluci s kytarou,

stojí v teskném zamyšlení a hosté mizí bez placení

a říkají si, že je zas pod párou.
\end{SBChorus}
\begin{SBVerse}
Až někdy navštívíte Vizovice, zaparkujte u silnice,

tož tam, co je tech nejvíc obchodů,

rozhlédněte se vpředu, vzadu, a, nemáte-li oční vadu,

ucítíte hospodu.

A když se zeptáte na hospodského, dovíte se od každého,

že je to notorický lhář, ale já ho \Ch{C}{viděl} jednou v \Ch{G/H}{květnu,}

jak tam \Ch{ami}{stál} a v ruce \Ch{D}{flétnu,} a kolem \Ch{G}{hlavy}, 

kolem hlavy vám měl svato\Ch{C}{zář.}
\end{SBVerse}
\begin{SBChorus}
\end{SBChorus}
\end{song}

\pagebreak
