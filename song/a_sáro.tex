\begin{song}{Sáro}{C}{Traband}

\begin{SBVerse}

Sbor \Ch{ami}{kajícných} \Ch{emi}{mnichů} \Ch{F}{jde} krajinou \Ch{C}{v }tichu 

a pro \Ch{F}{vše}chnu lidskou \Ch{C}{pýc}hu má jen \Ch{F}{přezíravý} \Ch{G}{smích.}

A z prohraných válek se vojska domů vrací 

však zbraně stále burácí a bitva zuří v nich.

\end{SBVerse}

\begin{SBChorus}

Sáro, Sáro, v noci se mi zdálo, 

že tři andělé boží k nám přišli na oběd. 

Sáro, Sáro, jak moc anebo málo 

mi chybí abych tvijí duši mohl rozumět.

\end{SBChorus}

\begin{SBVerse}

Vévoda v zámku čeká na balkóně 

až přivedou mu koně, pak mává na pozdrav. 

Srdcová dáma má v každé ruce růže, 

tak snadno pohřbít může sto urozených hlav.

\end{SBVerse}

\begin{SBVerse}

Královnin šašek s pusou od povidel 

sbírá zbytky jídel a myslí na útěk. 

A v podzemí skrytí slepí alchymisté 

už oběvili jistě proti povinnosti lék!

\end{SBVerse}

\begin{SBVerse}

Páv pod tvým oknem zpívá sotva procit 

o tajemstvích noci ve tvých zahradách. 

A já, potulný kejklíř, co svázali mu ruce, 

teď hraju o tvé srdce a chci mít tě nadosah!

\end{SBVerse}

\begin{SBChorus}

Sáro, Sáro, pomalu a líně 

s hlavou na tvém klíně chci se probouzet. 

Sáro, Sáro, Sáro, Sáro rosa padá ráno 

a v poledne už možná bude jiný svět!

\end{SBChorus}

\begin{SBVerse}

Sáro, Sáro, vstávej milá Sáro, 

\Ch{F}{andělé} k nám \Ch{dmi}{přišli} na \Ch{Cmaj}{oběd.}

\end{SBVerse}

\end{song}

\pagebreak