\setcounter{page}{50}
\begin{song}{Marsyas a Apollon}{G}{Marsyas}

\begin{SBVerse}

\Ch{G}{Ta} krásná dívka, \Ch{ami}{co} se bojí o \Ch{D}{svoji} krásu

\Ch{G}{Athéna} \Ch{emi}{jméno} má \Ch{C}{za} starých \Ch{G}{dávných} časů

\Ch{G}{odhodí} flétnu, \Ch{ami}{hrát} nejde s \Ch{D}{nehybnou} tváří

\Ch{G}{kdo} si ji \Ch{emi}{najde} dřív, \Ch{C}{tomu} se \Ch{G}{přání} zmaří.

\end{SBVerse}

\begin{SBChorus}

\Ch{emi}{Tak }si Marsyas \Ch{D}{mámen} flétnou \Ch{emi}{věří,} že \Ch{C}{musí} přetnout

$|$: \Ch{G}{jedno} pravidlo, \Ch{D}{sázku} a \Ch{C}{hrát} \Ch{G}{líp} \Ch{D}{než} \Ch{G}{bůh} :$|$

\end{SBChorus}

\begin{SBVerse}

Bláznivý nápad, snad nejvýš Marsyas míří

Apollón souhlasí, oba se s trestem smíří

král Midas má říct, kdo je lepši, Apollón zpívá

o život soupeří, jen jeden vítěz bývá

\end{SBVerse}

\begin{SBChorus}

Tak si Marsyas mámen flétnou věří a musí přetnout

$|$: jedno pravidlo, sázku a hrát líp než bůh :$|$

\end{SBChorus}

\begin{SBVerse}

Obrátí nástroj, už ví, že nebude chválen

prohrál a zápolí podveden vůlí krále

sám v tichém hloučku, sám na strom připraví ráhno

satyra k hrůze všech zaživa z kůže stáhnou

\end{SBVerse}

\begin{SBChorus}

Tak si Apollón změřil síly, každý se musel mýlit

$|$: Nikdo nemůže kouzlit a hrát líp než bůh :$|$

\end{SBChorus}

\begin{SBVerse}

Ta krásná dívka, co se bála o svoji krásu

dárkyně moudrosti za starých dávných časů

Teď v tichém hloučku, v jejích rukou úroda, spása

Athéna jméno má, chybí jí tvář a krása

\end{SBVerse}

\begin{SBChorus}

Dy dy dy \dots

\end{SBChorus}

\end{song}

\pagebreak
