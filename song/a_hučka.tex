\setcounter{page}{32}
\begin{song}{Hučka}{G}{Zelenáči}
\begin{SBVerse}
Hučku \Ch{G}{svou} na pozdrav smekám, světla \Ch{ami}{vlaků} vidím plát,

tak na \Ch{D}{svůj} nárazník čekám, už jsem \Ch{C}{tě} měl ako\Ch{G}{rát.}

Zejtra ráno, až se vzbudíš, zjistíš, že se slehla zem

a tvůj miláček že pláchnul půlnočním expresem.
\end{SBVerse}
\begin{SBChorus}
Za chví li už budu v dáli, za chví li mi bude fajn,

o tvý  lásce, která pálí, nebu du mít ani  šajn.

Za chví li už budu v dáli, za chví li mi bude fajn,

o tvý  lásce, která pálí, nebu du mít ani  šajn.
\end{SBChorus}
\begin{SBVerse}
Nejdřív zní vlakovej zvonec, pak píšťala, je mi hej,

konečně vím, že je konec naší lásce tutovej.

Z kapsy tahám harmoniku, tuláckej son budu hrát,

sedím si na nárazníku a je mi tak akorát. 
\end{SBVerse}
\begin{SBVerse}
Polámalse mraveneček, ví to celá obora,

o půlnoci zavolali mravnčího doktora.

Doktor klepe na srdíčko, potom píše recepis:

"třikrát ránu mezi voči, bude chlapík jako rys."

Dali ránu mezi oči, pohladili po čele,

bum a mrtvej mraveneček ráno leží v kostele.
\end{SBVerse}
\begin{SBVerse}
Pec nám spadla, pec nám spadla, kdopak nám ji postaví,

starej pecař není doma a mladej to neumí.

Zavoláme na dědečka, ten má velký kladivo,

dá do toho čtyři rány a už letí na pivo.
\end{SBVerse}
\begin{SBVerse}
Skálou, stepí, divočinou, hladový a roztrhán,

s ohněm v ruce, s nožem v srdci, znaven běží partyzán.
\end{SBVerse}
\end{song}

\pagebreak
