\setcounter{page}{86}
\begin{song}{Těšínská}{C}{Jaromír Nohavica}

\begin{SBVerse}

\Ch{ami}{Kdybych} se narodil \Ch{dmi}{před} sto léty \Ch{F}{}

\Ch{E7}{v} tomhle \Ch{ami}{městě.} \Ch{dmi,F,E7,ami} {}

\Ch{ami}{u Larichů} na zahradě \Ch{dmi}{trhal} bych květy\Ch{F}{}

\Ch{E7}{své} ne\Ch{ami}{věstě}.\Ch{dmi,F,E7,ami}{}

\Ch{C}{Moje} nevěsta by byla \Ch{dmi}{dcera} ševcova

\Ch{F}{z} domu Kamińskich \Ch{C}{od}někud ze Lvova

kochał bym ja i \Ch{dmi}{pieśćił} \Ch{F}{chy}\Ch{E7}{ba} lat \Ch{ami}{dwie}śćie.

\end{SBVerse}

\begin{SBVerse}

Bydleli bychom na Sachsenbergu v domě u žida Kohna.

Nejhezčí ze všech těšínských šperků byla by ona.

Mluvila by polsky a trochu česky,

pár slov německy a smála by se hezky.

Jednou za sto let zázrak se koná, zázrak se koná.

\end{SBVerse}

\begin{SBVerse}

Kdybych se narodil před sto léty byl bych vazačem knih.

U Prohazků dělal bych od pěti do pěti a sedm zlatek za to bral bych.

Měl bych krásnou ženu a tři děti,

zdraví bych měl a bylo by mi kolem třiceti,

celý dlouhý život před sebou celé krásné dvacáté století.

\end{SBVerse}

\begin{SBVerse}

Kdybych se narodil před sto léty v jinačí době

u Larichů na zahradě trhal bych květy má lásko tobě.

Tramvaj by jezdila přes řeku nahoru,

slunce by zvedalo hraniční závoru

a z oken voněl by sváteční oběd.

\end{SBVerse}

\begin{SBVerse}

Večer by zněla od Mojzese melodie dávnověká,

bylo by léto tisíc devět set deset za domem by tekla řeka.

Vidím to jako dnes šťastného sebe,

ženu a děti a těšínské nebe.

Jěště,že člověk nikdy neví co ho čeká.

\end{SBVerse}

\end{song}

\clearpage
