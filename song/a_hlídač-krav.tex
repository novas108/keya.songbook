\setcounter{page}{26}
\begin{song}{Hlídač krav}{C}{Jaromír Nohavica}
\begin{SBVerse}
\Ch{C}{Když} jsem byl malý, říkali mi naši:

Dobře se uč a jez chytrou kaši,

\Ch{F}{až} jednou vyrosteš, \Ch{G}{budeš} doktorem \Ch{C}{práv}.

Takový doktor si sedí pěkně v suchu,

bere velký peníze a škrábe se v uchu,

\Ch{F}{já} jim ale na to řek': Ch\Ch{G}{ci} být hlídačem \Ch{C}{krav.}
\end{SBVerse}
\begin{SBChorus}
Já chci \Ch{C}{mít} čapku s bambulí nahoře,

jíst kaštany a mýt se v lavoře,

\Ch{F}{od} rána po celý \Ch{G}{den} zpívat si \Ch{C}{jen,}

zpívat si: \Ch{C}{pam} pam pam\Ch{F}{ } \Ch{G}{\dots} \Ch{C}{ } 
\end{SBChorus}
\begin{SBVerse}
K vánocům mi kupovali hromady knih,

co jsem ale vědět chtěl, to nevyčet' jsem z nich:

nikde jsem se nedozvěděl, jak se hlídají krávy.

Ptal jsem se starších a ptal jsem se všech,

každý na mě hleděl jako na pytel blech,

každý se mě opatrně tázal na moje zdraví.
\end{SBVerse}
\begin{SBVerse}
Dnes už jsem starší a vím, co vím,

mnohé věci nemůžu a mnohé smím,

a když je mi velmi smutno, lehnu si do mokré trávy.

S nohama křížem a s rukama za hlavou

koukám nahoru na oblohu modravou,

kde se mezi mraky honí moje strakaté krávy.
\end{SBVerse}
\end{song}

\pagebreak
